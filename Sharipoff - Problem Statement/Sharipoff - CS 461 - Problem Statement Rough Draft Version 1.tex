\documentclass[onecolumn, draftclsnofoot,10pt, compsoc]{IEEEtran}
\usepackage{graphicx}
\usepackage{url}
\usepackage{setspace}

\usepackage{geometry}
\geometry{textheight=9.5in, textwidth=7in}

% 1. Fill in these details
\def \CapstoneTeamName{		Team #25}
\def \CapstoneTeamNumber{		25}
\def \GroupMemberOne{			Lazar Sharipoff}
\def \GroupMemberTwo{			Jordan Davis}
\def \GroupMemberThree{			Glen Anderson}
\def \CapstoneProjectName{		Smart Home Intercom System}
\def \CapstoneSponsorCompany{	Oregon State EECS}
\def \CapstoneSponsorPerson{		Kevin D McGrath}

% 2. Uncomment the appropriate line below so that the document type works
\def \DocType{		Problem Statement
				%Requirements Document
				%Technology Review
				%Design Document
				%Progress Report
				}
			
\newcommand{\NameSigPair}[1]{\par
\makebox[2.75in][r]{#1} \hfil 	\makebox[3.25in]{\makebox[2.25in]{\hrulefill} \hfill		\makebox[.75in]{\hrulefill}}
\par\vspace{-12pt} \textit{\tiny\noindent
\makebox[2.75in]{} \hfil		\makebox[3.25in]{\makebox[2.25in][r]{Signature} \hfill	\makebox[.75in][r]{Date}}}}
% 3. If the document is not to be signed, uncomment the RENEWcommand below
%\renewcommand{\NameSigPair}[1]{#1}

%%%%%%%%%%%%%%%%%%%%%%%%%%%%%%%%%%%%%%%
\begin{document}
\begin{titlepage}
    \pagenumbering{gobble}
    \begin{singlespace}
    	\includegraphics[height=4cm]{coe_v_spot1}
        \hfill 
        % 4. If you have a logo, use this includegraphics command to put it on the coversheet.
        %\includegraphics[height=4cm]{CompanyLogo}   
        \par\vspace{.2in}
        \centering
        \scshape{
            \huge CS Capstone \DocType \par
            {\large\today}\par
            \vspace{.5in}
            \textbf{\Huge\CapstoneProjectName}\par
            \vfill
            {\large Prepared for}\par
            \Huge \CapstoneSponsorCompany\par
            \vspace{5pt}
            {\Large\NameSigPair{\CapstoneSponsorPerson}\par}
            {\large Prepared by }\par
            Group\CapstoneTeamNumber\par
            % 5. comment out the line below this one if you do not wish to name your team
            \CapstoneTeamName\par 
            \vspace{5pt}
            {\Large
                \NameSigPair{\GroupMemberOne}\par
                \NameSigPair{\GroupMemberTwo}\par
                \NameSigPair{\GroupMemberThree}\par
            }
            \vspace{20pt}
        }
        \begin{abstract}
        % 6. Fill in your abstract    
        	This document is written using one sentence per line.
        	This allows you to have sensible diffs when you use \LaTeX with version control, as well as giving a quick visual test to see if sentences are too short/long.
        	If you have questions, ``The Not So Short Guide to LaTeX'' is a great resource (\url{https://tobi.oetiker.ch/lshort/lshort.pdf})
        \end{abstract}     
    \end{singlespace}
\end{titlepage}
\newpage
\pagenumbering{arabic}
\tableofcontents
% 7. uncomment this (if applicable). Consider adding a page break.
%\listoffigures
%\listoftables
\clearpage

% 8. now you write!
\section{Problem Description}
Traditional home intercom systems work by pressing a button, 
and every speaker in the every room becomes active.
When the button is pressed on "Smart Home Intercom System" the only devices that receive a signal are the ones in rooms containing other people. When a device is signaled it prompts to trade video and/or audio.

The system should support a minimum of 4 devices with a: touch screen, camera, microphone, and speaker built into each device. 
The connection between each device must be secure, with absolutely no outside connection. The video and audio must also be securely encrypted and streamed only (no recording). The system should have an "admin" mode that allows the user view the cameras of every device that is connected; and a "Guest" mode that allows calls to be made to the device, but unable to make calls without some sort of authorization. 
For instance, a call could be made to a device on the porch, but the device on the porch would not be able to make a call any devices inside. 
The "Smart Home Intercom System" should run out of the box and have "plug and play" functionality for the hardware. 
The system must be able to be expanded easily by connecting a new device and being fully functional from then on.

The system could have a backup battery as a potential add-on in case of loss of power,
and the possible functionality to display weather data or appointments for the day.



\section{Problem Solution}
The "Smart Home Intercom System" will have a base hardware of either a Raspberry Pi 3 or Zero W depending on the required computing power. 
Each device will have a 2D camera, microphone, speaker, and a sensible to use interface via touchscreen. 
The connection between devices will be ad hoc, and could be implemented using BATMAN with a pin required to connect to the system.
"Admin" mode will be available from any device with a security conformation, and might have the ability to designate other devices as "Guest" devices. 
Otherwise each device would have to be manually put into "Guest" mode.
Each device will be able to be mounted to the wall with the camera height placement being solved at installation.

The system will have the functionality of remembering what room a person was last in and will only detect humans ( no cats or other things).
In the case that no one is in view of a camera will be determined by the user by presenting two options.
The first option being that the user speaks out to all the other devices.
The second option being that the user selects which room to speak to, with the information of which room someone was last in.

The backup battery could be implemented by plugging the device into a battery bank, and then keeping the battery bank plugged into an outlet. The ability to display the weather and daily appointments while still maintaining a secure connection might be solved with a device being paired to a phone through Bluetooth, or a Home WI-Fi system.


\section{Performance Metrics}
We will know that our project is completed by completing the following goals: 
The "Smart Home Intercom System" is able to be pulled out of the box and has a sensible to use interface that allows to be set up by my mother without hassle. 
The system supports 4 devices minimum with easy expansion by connecting a new device to the system using a required pin, while being closed off from outside connection.
The system will only stream video and audio across devices securely through encryption so that the feeds wont be able to be viewed from an outside source.
"Admin" mode is able to view the camera feed of every device.
"Admin" mode functionality is accessible only after a security confirmation.
"Guest" mode is only able to receive calls until is authorized by a non "Guest" mode device to make a call.
Each device is able to be mounted to the wall.

If the backup battery is implemented then we will know it is completed by being able to connect a device to the battery and it will run at least two hours without being plugged into the wall. 
If the weather and daily appointments functionality is implemented, 
then we will know that it is completed by being able to connect a phone and having a device in the system display the desired information on the screen.


\end{document}