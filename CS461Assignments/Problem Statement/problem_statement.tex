\documentclass[onecolumn, draftclsnofoot,10pt, compsoc]{IEEEtran}
\usepackage{graphicx}
\usepackage{url}
\usepackage{setspace}

\usepackage{geometry}
\geometry{textheight=9.5in, textwidth=7in}

% 1. Fill in these details
\def \CapstoneTeamName{		Team 25}
\def \CapstoneTeamNumber{		25}
\def \GroupMemberOne{			Lazar Sharipoff}
\def \GroupMemberTwo{			Jordan Davis}
\def \GroupMemberThree{			Glen Anderson}
\def \CapstoneProjectName{		Smart Home Intercom System}
\def \CapstoneSponsorCompany{	Oregon State EECS}
\def \CapstoneSponsorPerson{		Kevin D McGrath}

% 2. Uncomment the appropriate line below so that the document type works
\def \DocType{		Problem Statement
				%Requirements Document
				%Technology Review
				%Design Document
				%Progress Report
				}
			
\newcommand{\NameSigPair}[1]{\par
\makebox[2.75in][r]{#1} \hfil 	\makebox[3.25in]{\makebox[2.25in]{\hrulefill} \hfill		\makebox[.75in]{\hrulefill}}
\par\vspace{-12pt} \textit{\tiny\noindent
\makebox[2.75in]{} \hfil		\makebox[3.25in]{\makebox[2.25in][r]{Signature} \hfill	\makebox[.75in][r]{Date}}}}
% 3. If the document is not to be signed, uncomment the RENEWcommand below
%\renewcommand{\NameSigPair}[1]{#1}

%%%%%%%%%%%%%%%%%%%%%%%%%%%%%%%%%%%%%%%
\begin{document}
\begin{titlepage}
    \pagenumbering{gobble}
    \begin{singlespace}
    	\includegraphics[height=4cm]{coe_v_spot1}
        \hfill 
        % 4. If you have a logo, use this includegraphics command to put it on the coversheet.
        %\includegraphics[height=4cm]{CompanyLogo}   
        \par\vspace{.2in}
        \centering
        \scshape{
            \huge CS Capstone \DocType \par
            {\large\today}\par
            \vspace{.5in}
            \textbf{\Huge\CapstoneProjectName}\par
            \vfill
            {\large Prepared for}\par
            \Huge \CapstoneSponsorCompany\par
            \vspace{5pt}
            {\Large\NameSigPair{\CapstoneSponsorPerson}\par}
            {\large Prepared by }\par
            Group\CapstoneTeamNumber\par
            % 5. comment out the line below this one if you do not wish to name your team
            %\CapstoneTeamName\par 
            \vspace{5pt}
            {\Large
                \NameSigPair{\GroupMemberOne}\par
                \NameSigPair{\GroupMemberTwo}\par
                \NameSigPair{\GroupMemberThree}\par
            }
            \vspace{20pt}
        }
        \begin{abstract}
        % 6. Fill in your abstract    
        	A home intercom system allows individuals to communicate with each other between rooms. This system can be made more intelligent by adding cameras, microphones and functionality that allows users to specify rooms to page and know where other users are located. This system will be implemented with Raspberry Pi 3 devices as they allow for enough processing power to run the system but are also compact, which will keep the system usable and out of the way. The resulting system should be able to work out of the box and allow for full user functionality.
        \end{abstract}     
    \end{singlespace}
\end{titlepage}
\newpage
\pagenumbering{arabic}
\tableofcontents
% 7. uncomment this (if applicable). Consider adding a page break.
%\listoffigures
%\listoftables
\clearpage

% 8. now you write!
\section{Problem Description}
Communication in homes is commonly done by shouting from one room to another or walking from room to room. A step forward from that was the implementation of intercommunicating telephones where one could place a call to other rooms with receivers. Modern times updated that to allow for a paging system where someone can press a call button and it would broadcast to all available receivers within a household. This method of all or nothing communication simply creates noise pollution throughout a household, interrupting everyone who has nothing to do with what is going on over the intercom. A smart home intercom system would allow for advanced technologies and products to be integrated together and prevent undesired noise pollution with directed paging. Installation of previous iterations of intercom systems has always been a hassle. Running wires and power lines between units that are desired to be connected costs time and money. Even adding a unit into existing installations might require a couple hours fishing lines through a room. A smart system could integrate with itself wirelessly, only requiring to be plugged into power. Beyond that, there would be few limitations on adding more units. Ultimately communication within a home is dated and very little has been done to bring it into the digital age.

Existing smart home intercom systems rely on a WiFi connection to function, which is not always convenient and presents a security issue if the network is compromised. For example, the Nucleus WiFi Intercom system allows users to place and receive calls within a home or between homes. However, the system’s connection to external WiFi and applications could compromise the security of a user's data.  




\section{Problem Solution}
A smart home intercom system should be simple. It should be able to be taken out of the box and able to run without requiring extensive user configuration. Each unit should be comprised of a wall mountable case with a touchscreen that houses a Raspberry Pi and has a camera, microphone, and speaker attached to it. The software should be integrated onto the unit already and it should require a PIN or password to connect to the existing systems and then ask for a room designation and work as intended. The unit should display a user interface that allows for easy access to controls for the speaker volume, microphone feedback, and brightness. The interface should display a button to place a call and an option to view video feed from a selected room on the left side of the screen. There should also be a menu option to allow for a user to change settings, such as setting either visitor or administrator mode. Administrator mode would allow for a user to view all connected units camera feeds and select audio feeds, whereas visitor mode would disable video feedback and only allow for paging to select rooms.

The system should be able to track rooms that are currently occupied and keep a record of where a user was last if they are not detected. It should be able to identify a person entering within view of the camera and specify that person’s occupancy of a room or last known room location. The screen should turn off after no one has been detected for a certain amount of time and it should adjust its own brightness in ambient light. The feeds between units should be on a secure, closed network to ensure that the units are unable to communicate over the internet.

There is potential for a specific unit to be able to sync to a person’s phone and retrieve calendar and weather updates and alert the user of whatever those might be, but this would need to be implemented in a way that doesn’t compromise the overall security of the system. There is also potential to add a secondary power supply to the Raspberry Pi devices such as a battery pack but this will depend on the power requirements for the system.




\section{Performance Metrics}

\begin{itemize}
\item The absolute measurement for this project is four functioning prototypes as described by below specifications.
\item It should be scaleable amongst each other and it should allow for swapping out nodes.
\item Initial prototyping involves configuring microphone, video, speaker and touchscreen systems with the Raspberry Pi and then implementing programs to further their usability.
\item Each unit should be able to make and receive audio calls between each other and allow for video to be streamed.
\item Devices should notify one another of the occupancy to allow for those units to direct a call correctly.
\item Ultimately the unit should be able to be consolidated and packaged within a box and then booted to immediately be part of a large system of devices after a PIN or password verification.
\item The system should enforce a visitor and administrator mode as discussed above.
\end{list}









\end{document}