\documentclass[onecolumn, draftclsnofoot,10pt, compsoc]{IEEEtran}
\usepackage{graphicx}
\usepackage{url}
\usepackage{setspace}
\usepackage{pgfgantt}
\usepackage{textcomp}

\usepackage{geometry}
\geometry{textheight=9.5in, textwidth=7in}

% 1. Fill in these details
\def \CapstoneTeamName{		}
\def \CapstoneTeamNumber{		25}
\def \GroupMemberOne{			Lazar Sharipoff}
\def \GroupMemberTwo{			}
\def \GroupMemberThree{			}
\def \CapstoneProjectName{		Smart Home Intercom System}
\def \CapstoneSponsorCompany{	Oregon State EECS}
\def \CapstoneSponsorPerson{		D. Kevin McGrath}

% 2. Uncomment the appropriate line below so that the document type works
\def \DocType{		%Problem Statement
				%Requirements Document
				%Technology Review
				Design Document
				%Progress Report
				}
\newcommand\tab[1][1cm]{\hspace*{#1}}			
\newcommand{\NameSigPair}[1]{\par
\makebox[2.75in][r]{#1} \hfil 	\makebox[3.25in]{\makebox[2.25in]{\hrulefill} \hfill		\makebox[.75in]{\hrulefill}}
\par\vspace{-12pt} \textit{\tiny\noindent
\makebox[2.75in]{} \hfil		\makebox[3.25in]{\makebox[2.25in][r]{Signature} \hfill	\makebox[.75in][r]{Date}}}}
% 3. If the document is not to be signed, uncomment the RENEWcommand below
\renewcommand{\NameSigPair}[1]{#1}

%%%%%%%%%%%%%%%%%%%%%%%%%%%%%%%%%%%%%%%
\begin{document}

\clearpage

% 8. now you write!
\section{Video Display Hardware/Software, and Network Encryption}


\subsection{Video Display Hardware}
The system will use a 7" display to show the UI and incoming/outgoing video feed during video calls. 
The display will connect to an adapter board with a ribbon cable, and Raspberry Pi 3 (RPi3) then connects to the adapter board with another ribbon cable and 2 jumper wires connected to the respective pins. 
Jumper wire one is connected to PIN 2 on the RPi3, and the 5V PIN on the adapter board. Jumper wire two is connected to PIN 6 on the RPi3, and the GND PIN on the adapter board. Then the display, adapter board, and RPi3 are housed in a wall mountable case. The display hardware is plug and play with the RPi3 so there is no additional configuration required for the display to show the UI, however,  incoming/outgoing video feeds need to be handled by additional software before being able to be displayed.


\subsection{Video Display Software}
VLC will be used to handle the incoming/outgoing video feeds before being displayed on the screen. 
Installation of the software onto the system is done through a package, because the system is using Raspbian as the OS and VLC supports Debian Linux distributions.
The software does not need additional configuration to be able to playback video, however, it does need to be configured to be able to stream video feed between two nodes. 
Configuration of streaming video is done by setting the first node to stream the video feed out, then the second node connects to and displays the stream with VLC using the IP address of the first node. 
The configurations are required to be set on both nodes, so that the video feeds of both nodes can be displayed on each other, and must be set every time a video call is attempted to be established.


\subsection{Network Encryption}
To protect users video and audio feeds an encryption method needs to be implemented at the application level. 
Salsa20 satisfies this requirement and works in conjunction with VLC by encrypting the video/audio feed as it streams.
To test that the video and/or audio feed between two nodes is correctly encrypted, the feed can be intercepted by a third node, and if the video and audio stream is unable to be viewed by that node then the feed is correctly encrypted.




\end{document}
